\chapter{Introduction}
\label{introduction}

The Higgs boson is often described as one of the cornerstones of the Standard Model. Since the conception of the Higgs mechanism as the source of electroweak symmetry breaking in the early 1960s, countless collider experiments have searched for this elusive particle. This dissertation presents multiple studies of the Higgs boson with the ATLAS detector at the Large Hadron Collider (LHC). 

One of the first priorities of the early LHC was the search for the Higgs boson. This search was first tackled in three main channels: $H\to \gamma\gamma$, $H\to ZZ$, and $H\to WW$. Each channel has its own merits, but the $WW$ is particularly suited to searching over a wide range of masses. The $H\to WW$ branching ratio is large and it is the primary decay channel above the $2m_W$ mass threshold. 

While the rate of events produced in $H\to WW$ is large, the channel poses some challenges. First, the most common mode of study for this channel is \HWWfull. With neutrinos in the final state, it is not possible to fully reconstruct the invariant mass of the parent Higgs like the $\gamma\gamma$ and $ZZ\to 4\ell$ channels. Second, the final state topology is mimicked by a wide variety of backgrounds that need to be properly estimated. This means tailored selection requirements for background reduction and robust background estimation techniques must both be developed. 

In 2012, the ATLAS and CMS experiments announced the discovery of a new particle consistent with the Higgs boson~\cite{Discovery,CMSDiscovery}. In ATLAS, this discovery was made with $4.8 \ifb$ collected at $\sqrt{s} = 7 \TeV$ and $5.8 \ifb$ at $\sqrt{s} = 8 \TeV$. The \HWWfull analysis played an important role in this discovery. After the discovery, measurement of the properties of the newly discovered particle and confirmation of its consistency with the Standard Model Higgs were the main priorities. The $WW$ channel is also uniquely suited to these types of measurements. Because of its good rate, it offers some of the best cross section measurements available among the various Higgs decay modes. It is also suited for measurement of multiple Higgs production modes, like the vector boson fusion (VBF) mode, where incoming quarks radiate $W/Z$ bosons which fuse to make a Higgs. In VBF production with the $WW$ decay channel, the coupling of the Higgs to the $W$ boson is present in both the production and decay which allows for more precise measurements of this coupling than other channels which rely on gluon fusion production (where gluons couple to the Higgs through a top loop in the production). The measurement of VBF carries the additional challenge that its cross section is an order of magnitude smaller than that of gluon fusion, meaning that the large branching ratio to $WW$ offers an additional advantage in isolating this production mode. In the final ATLAS Run 1 results, combining $20.3 \ifb$ taken at $\sqrt{s} = 8 \TeV$ with the $4.8 \ifb$ collected at $\sqrt{s} = 7 \TeV$, the $WW$ channel achieved its first observation of VBF production of the Higgs.

After Run $1$ of the LHC, with the existence of the Higgs now firmly established, the focus shifted to searches for physics beyond the Standard Model. In particular, searches for high mass resonances benefit from the LHC's increase to $\sqrt{s} = 13 \TeV$ in Run 2. The newly discovered Higgs can be used as a tool in these searches. Higgs pair production in the Standard Model has a low cross section that requires large datasets (on the order of the LHC's lifetime) for full measurement. However, new physics can modify this cross section, especially new resonances which decay to two Higgs bosons. A search for Higgs pair production in the $HH\to b\bar{b}b\bar{b}$ final state was performed with $3.2 \ifb$ collected with ATLAS at $\sqrt{s} = 13 \TeV$ in 2015. 

This dissertation begins by discussing the discovery of the Higgs and the role of the \HWWfull channel. It then discusses the first observation of the VBF production mode in \HWWfull with the full ATLAS Run 1 dataset, as well as the final combined Run 1 measurements from this channel. Finally, it presents a search for Higgs pair production in the $HH\to b\bar{b}b\bar{b}$ channel. It is organized into four parts. 

Part 1 presents the theoretical and experimental background required for the subsequent parts. Chapter 1 gives an overview of Higgs physics, particularly single and double Higgs production in the Standard Model and beyond. Chapter 2 presents details regarding the Large Hadron Collider and the ATLAS experiment. The evolution of machine conditions, descriptions of the ATLAS sub-detectors, and an overview of object reconstruction in ATLAS are all shown. A brief interlude on the ATLAS Muon New Small Wheel upgrade is also given, as this upgrade has been a focus of my graduate work and will have important impact on ATLAS' ability to study the Higgs at the High Luminosity LHC. 

Part 2 discusses the observation and measurement of the Higgs in the \HWWfull channel in the ATLAS Run 1 dataset at $\sqrt{s} = 7$ and $8 \TeV$. Because I worked in this channel from before the discovery through to the final analysis of the Run 1 dataset, Part 2 is organized in such a way to allow easy presentation of multiple analyses on different subsets of the full Run 1 dataset. Chapter 3 presents a general overview of the $\HWW$ analysis strategy and defines many of the variables and common elements used in the rest of Part 2. Chapter 4 presents the discovery of the Higgs boson, focusing on the role of the $WW^*$ channel in this discovery. Chapter 5 presents the first observation of the VBF production mode of the Higgs in the $WW^*$ channel, a study which was done on the full Run 1 ATLAS dataset. In this chapter, the focus is mainly on the selection cut-based VBF analysis. The cut-based analysis was an important first step to the final VBF result which used a Boosted Decision Tree (BDT). Where appropriate, connections between the cut-based and BDT analyses are shown and their compatibility is discussed. Finally, the VBF analysis was an important input into the combined Run 1 \HWWfull result, which used both the gluon fusion and VBF channels in a combined fit to infer properties of the Higgs, including its couplings to the gauge bosons and its production cross section. This is the topic of Chapter 6. 

Part 3 presents a search for Higgs pair production in the $HH \to b\bar{b} b\bar{b}$ channel. Chapter 7 presents an overview of this search in the boosted regime, where the Higgs pairs are the result of the decay of a heavy resonance. Chapter 8 shows the combined results between the boosted regime and the resolved regime, which is sensitive to lower mass resonances and non-resonant Higgs pair production. Finally, Part 4 presents a conclusion and brief outlook of future Higgs physics with ATLAS.