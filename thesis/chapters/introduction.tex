\chapter{Introduction}
\label{introduction}

The Higgs boson is often described as one of the cornerstones of the Standard Model. Since the conception of the Higgs mechanism as the source of electroweak symmetry breaking in the early 1960s, countless collider experiments have searched for this elusive particle. This dissertation presents multiple studies of the Higgs boson with the ATLAS detector at the Large Hadron Collider (LHC). It is organized into four parts. 

Part 1 presents the theoretical and experimental background required for the subsequent parts. Chapter 1 gives an overview of Higgs physics, particularly single and double Higgs production in the Standard Model and beyond. Chapter 2 presents details regarding the Large Hadron Collider and the ATLAS experiment. The evolution of machine conditions, descriptions of the ATLAS sub-detectors, and an overview of object reconstruction in ATLAS are all shown. A brief interlude on the ATLAS Muon New Small Wheel upgrade is also given, as this upgrade has been a focus of my graduate work and will have important impact on ATLAS' ability to study the Higgs at the High Luminosity LHC. 

Part 2 discusses the observation and measurement of the Higgs in the \HWWfull channel in the ATLAS Run 1 dataset at $\sqrt{s} = 7$ and $8 \TeV$. Because I worked in this channel from before the discovery through to the final analysis of the Run 1 dataset, Part 2 is organized in such a way to allow easy presentation of multiple analyses on different subsets of the full Run 1 dataset. Chapter 3 presents a general overview of the $\HWW$ analysis strategy and defines many of the variables and common elements used in the rest of Part 2. Chapter 4 presents the discovery of the Higgs boson, focusing on the role of the $WW^*$ channel in this discovery. Chapter 5 presents the first observation of the vector boson fusion (VBF) production mode of the Higgs in the $WW^*$ channel, a study which was done on the full Run 1 ATLAS dataset. In this chapter, the focus is mainly on the selection cut-based VBF analysis. The cut-based analysis was an important first step to the final VBF result which used a Boosted Decision Tree (BDT). Where appropriate, connections between the cut-based and BDT analyses are shown and their compatibility is discussed. Finally, the VBF analysis was an important input into the combined Run 1 \HWWfull result, which used both the gluon fusion and VBF channels in a combined fit to infer properties of the Higgs, including its couplings to the gauge bosons and its production cross section. This is the topic of Chapter 6. 

After the discovery of the Higgs and measurement of many of its properties in Run 1, it is natural to come up with ways that the Higgs could be used as a tool to search for physics beyond the Standard Model (BSM). One particular channel where this is possible is in a search for Higgs pair production, a process that can be enchanced by BSM physics. Part 3 presents a search for Higgs pair production in the $HH \to b\bar{b} b\bar{b}$ channel. Chapter 7 presents an overview of this search in the boosted regime, where the Higgs pairs are the result of the decay of a heavy resonance. Chapter 8 shows the combined results between the boosted regime and the resolved regime, which is sensitive to lower mass resonances and non-resonant Higgs pair production. 

Finally, Part 4 presents a conclusion and brief outlook of future Higgs physics with ATLAS.