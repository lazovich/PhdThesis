\chapter{Conclusion}
\label{conclusion}

After being sought for many years at different collider experiments, the Higgs boson was discovered by the ATLAS and CMS experiments in 2012, confirming the leading theory for the source of electroweak symmetry breaking and filling in the last missing piece of the Standard Model. After its discovery, measurements of the particle's detailed properties and searches for new particles decaying to Higgs final states were both extremely important in constraining physics beyond the Standard Model. This dissertation presented this evolution through two results: the observation and measurement of the Higgs boson in the \HWWfull channel at $\sqrt{s} = 7 \TeV$ and $\sqrt{s} = 8 \TeV$ and a search for Higgs pair production in the $HH\to b\bar{b}b\bar{b}$ channel at $\sqrt{s} = 13 \TeV$ with the ATLAS detector in $pp$ collisions at the Large Hadron Collider.

In the \HWWfull, results from both the discovery of the Higgs boson and the full ATLAS Run 1 dataset were presented. The Higgs boson was discovered with a $6.1\sigma$ significance in a combination of the $H\to\gamma\gamma$, $H\to ZZ 4\ell$, \HWWfull with $4.2\ifb$ at $\sqrt{s} = 7 \TeV$ and $5.2\ifb$ at $\sqrt{s} = 8 \TeV$. With the full $20.3\ifb$ at $\sqrt{s} =  8 \TeV$ and $4.2\ifb$ at $\sqrt{s} = 7 \TeV$, ATLAS achieved discovery level significance in the $\HWW$ channel alone and obtained the first observation of vector boson fusion production in that channel. The combined signal strength is measured to be $\mu = 1.09^{+0.23}_{-0.21}$. The total observed significance of the $\HWW$ process is observed to be $6.1\sigma$ (with $5.8 \sigma$ expected). Advanced methods for background reduction and estimation, particularly in same-flavor lepton final states, are shown. The VBF signal strength is measured to be $\mu_{\rm VBF} = 1.27^{+0.53}_{-0.45}$ with an observed significance of $3.2\sigma$ (with $2.7 \sigma$ expected). 

These results required many novel innovations. The increase of pileup interactions in the higher instantaneous luminosity LHC conditions of 2012 led to a degradation of missing transverse momentum resolution. As a result, the prominent $\ZDY$+jets background of the same flavor \HWWfull final states increased greatly. New variables, including a track-based missing transverse momentum and a measurement of the balance between the dilepton system and recoiling jets, allowed for significant reduction of this background. In the VBF channel, selections were optimized to exploit the unique VBF final state topology. Incorporating these variables into a boosted decision tree technique allowed the analysis to exceed the $3\sigma$ observation threshold.

At $\sqrt{s} = 13 \TeV$, a search for Higgs pair production in the $b\bar{b}b\bar{b}$ final state with $3.2\ifb$ was conducted. A signal region optimized for the boosted final states arising from high mass resonances was constructed. This signal region utilized large-radius calorimeter jets and $b$-tagging with small radius track jets to maximize the signal acceptance. No significant excesses were observed, and upper limits on cross sections are placed for spin-2 Randall Sundrum gravitons (RSG) and narrow spin-0 resonances. The increase in center of mass energy in Run 2 allowed this analysis to extend upper limits up to $3 \TeV$, while previous results from ATLAS in Run 1 only quotes limits up to $2 \TeV$. The cross section of $\sigma(pp \to \Gkk \to hh \to b\bar{b}b\bar{b})$ with $k/\bar{M}_{\rm Pl}=1$ is constrained to be less than $70 \fb$ for masses in the range $600 < m_{\Gkk} < 3000 \GeV$. For the RSG model with $k/\bar{M}_{\rm Pl}=2$, cross sections limits between $40\fb$ and $200 \fb$ are set for the mass range of $500 < m_{\Gkk} < 3000 \GeV$. The cross section upper limits for $\sigma(pp \to H \to hh \to b\bar{b}b\bar{b})$ ranges from $30$ to $300 \fb$ in the mass range of $500 < m_{H} < 3000 \GeV$. 



