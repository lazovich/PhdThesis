\chapter{Introduction}
\label{introduction}

The Higgs boson is often described as one of the cornerstones of particle physics. When the Standard Model was first developed as a theory to describe the fundamental particles and forces of nature, physicists were faced with a dilemma. The electroweak theory beautifully characterized both electromagnetism and the weak force with a single underlying framework. However, the mass of the weak $W$ and $Z$ bosons was puzzling given the fact that their electromagnetic counterpart, the photon, is massless. The Higgs mechanism was developed as the leading theory for the origin of this electroweak symmetry breaking. It predicted the existence of an additional spin zero boson in the Standard Model, the Higgs boson. Generations of collider experiments searched for this elusive particle. This dissertation presents research work on the Higgs boson from its discovery to its use as a tool in the search for physics beyond the Standard Model with the ATLAS detector at the Large Hadron Collider (LHC). 

One of the first priorities for the LHC when it began colliding proton beams in $2010$ was the search for the Higgs boson. This search was initially tackled in the \HWWfull channel, followed by the $H\to \gamma\gamma$ and $H\to ZZ^* \to 4\ell $ channels. Each channel has its own merits, but the $WW^*$ mode is particularly suited to searching over a wide range of masses. The $H\to WW^*$ branching ratio is large and it is the primary decay channel above the $2m_W$ mass threshold. Despite the fact that the full Higgs invariant mass cannot be reconstructed in the \HWWfull channel, its signal to background ratio makes it ideal for measurement of Higgs properties such as the production cross section and couplings. 

In 2012, the ATLAS and CMS experiments announced the discovery of a new particle consistent with the Higgs boson~\cite{Discovery,CMSDiscovery}. In ATLAS, this discovery was made with $4.8 \ifb$ collected at $\sqrt{s} = 7 \TeV$ and $5.8 \ifb$ at $\sqrt{s} = 8 \TeV$. This dissertation first presents the search for gluon fusion production of the Higgs in the \HWWfull channel, which played an important role in this discovery. Selection requirements which were optimized to maximize the discovery significance in this channel, as well as background estimation procedures, are discussed. 

After its discovery, interest in the Higgs shifted to focus on the measurement of its properties. As a result, extensions of the initial discovery analysis in larger datasets had two main goals. Improvement of signal to background ratio was important to allow for precision measurements. Also, searches for production modes of the Higgs with lower cross sections than gluon fusion were a priority. The first such extension presented in this dissertation is a tailored selection for $\ell\nu\ell\nu$ final states with same flavor leptons. Novel variables for the reduction of the $Z$+jets background that remain robust under increasing LHC instantaneous luminosities are shown. The second post-discovery result shown is the first evidence of Vector Boson Fusion (VBF) production of the Higgs boson. 

VBF production of the Higgs boson is particularly interesting in the \HWWfull final state. In this combination of production and decay modes, the Higgs boson couples exclusively to vector bosons, allowing for precise measurement of the Higgs-$W$ coupling constant. However, it is challenging to observe VBF Higgs production because its cross section at the LHC is an order of magnitude lower than gluon fusion production. The large $\HWW$ branching ratio thus presents another advantage over other final states. VBF production of the Higgs boson also creates two forward jets in addition to the Higgs, and these jets can be used to isolate VBF Higgs events from other production modes. The VBF \HWWfull analysis first created a selection requirement based signal region using variables constructed specifically for the VBF Higgs production topology. This ``cut-based" analysis is presented in detail in this dissertation. The VBF topology variables, once validated in the cut-based analysis, were then input into a multivariate boosted decision tree discriminant to achieve the first evidence of VBF Higgs production with the full $20.3 \ifb$ of $\sqrt{s} = 8 \TeV$ data in ATLAS. Combining these results with the dedicated gluon fusion Higgs production analysis allowed for precise measurement of the Higgs couplings. 

After a two year shutdown, the LHC restarted in 2015 with a center of mass energy of $\sqrt{s} = 13 \TeV$. This increase improved the LHC's ability to probe for physics beyond the Standard Model, and the Higgs sector remained one of the largest regions of unprobed phase space where such new physics could be discovered. Production of high mass resonances benefits most from the center of mass energy increase. In particular, the cross section for a generic gluon-initiated $2 \TeV$ resonance increased tenfold with the increase from $8$ to $13 \TeV$. Therefore, a natural next step in studies of the Higgs was a search for a new heavy resonance which decays into a pair of Higgs bosons. The final result shown in this dissertation is a search for resonant di-Higgs production in the $\FourBfull$ final state with $3.2 \ifb$ recorded by ATLAS at $\sqrt{s} = 13 \TeV$. This search has the unique advantage that it can both probe new physics and gain further understanding of the Higgs potential through constraints on SM pair production of the Higgs. It also extends the previous ATLAS results at $\sqrt{s} = 8 \TeV$ and probes higher mass resonances that were not previously accessible. It is an informative precursor to di-Higgs analyses at the future High Luminosity LHC (HL-LHC), where a projected dataset of $3000 \ifb$ at $\sqrt{s} = 14 \TeV$ will begin to become sensitive to the SM Higgs self coupling. 

As mentioned above, this dissertation begins by discussing the discovery of the Higgs and the role of the \HWWfull channel. It then presents the first evidence for the VBF production mode using the \HWWfull channel with the full ATLAS Run 1 dataset. It also shows the final combined Run 1 measurements of gluon fusion Higgs production from this channel. Finally, it presents a search for Higgs pair production in the $HH\to b\bar{b}b\bar{b}$ channel. It is organized into four parts. 

Part 1 presents the theoretical and experimental background required for the subsequent parts. Chapter 1 gives an overview of Higgs physics, particularly single and double Higgs production in the Standard Model and beyond. Chapter 2 presents details regarding the Large Hadron Collider and the ATLAS experiment. The evolution of machine conditions, descriptions of the ATLAS sub-detectors, and an overview of object reconstruction in ATLAS are all shown. A brief interlude on the ATLAS Muon New Small Wheel upgrade is also given, as this upgrade has been a focus of my graduate work and will have an important impact on ATLAS' ability to study the Higgs at the High Luminosity LHC. 

Part 2 discusses the observation and measurement of the Higgs in the \HWWfull channel in the ATLAS Run 1 dataset at $\sqrt{s} = 7$ and $8 \TeV$. Because I worked in this channel from before the discovery through to the final analysis of the Run 1 dataset, Part 2 is organized in such a way as to allow easy presentation of multiple analyses on different subsets of the full Run 1 dataset. Chapter 3 presents a general overview of the $\HWW$ analysis strategy and defines many of the variables and common elements used in the rest of Part 2. Chapter 4 presents the discovery and subsequent measurements of the Higgs boson, focusing on the role of the $WW^*$ channel in this discovery. Chapter 5 presents the first evidence for the VBF production mode of the Higgs, a result from the $WW^*$ channel in the full Run 1 ATLAS dataset. In this chapter, the focus is mainly on the cut-based VBF analysis. The cut-based analysis was an important first step to the final VBF result which used a boosted decision tree. Where appropriate, connections between the cut-based and BDT analyses are shown and their compatibility is discussed. Finally, the VBF analysis was an important input into the combined Run 1 \HWWfull result, which used both the gluon fusion and VBF channels in a combined fit to infer properties of the Higgs, including its couplings to the gauge bosons and its production cross section. This is the topic of Chapter 6. 

Part 3 presents a search for Higgs pair production in the $HH \to b\bar{b} b\bar{b}$ channel. Chapter 7 presents an overview of this search in the boosted regime, where the Higgs pairs are the result of the decay of a heavy resonance. Chapter 8 shows the combined results between the boosted regime and the resolved regime, which is sensitive to lower mass resonances and non-resonant Higgs pair production. Part 4 presents a conclusion and brief outlook of future Higgs physics with ATLAS.