\begin{savequote}[75mm]
The imagination of nature is far, far greater than the imagination of man.
\qauthor{Richard Feynman}
\end{savequote}

\chapter{Observation of Vector Boson Fusion production of $H\rightarrow WW^{*}\rightarrow \ell\nu\ell\nu$}

\section{Introduction}

After the discovery of a particle consistent with the Higgs boson, the $\HWW$ analysis had two main goals. The first goal was to increase the sensitivity of the analysis to fully confirm that the $\HWW$ process did indeed exist. The second goal was to characterize the particle as much as possible, including searching for the lower cross-section production modes, in order to confirm that it was indeed a Higgs boson.   This chapter presents a dedicated search for Vector Boson Fusion (VBF) production of a Higgs boson decaying via the \HWWfull mode. First, basics of the topology of VBF production are presented. Then, the details of the analysis are shown, including signal region definition, background estimation techniques, and systematic uncertainties. Finally, the results of the analysis are shown. As will be shown, this analysis is the first and most sensitive observation of the VBF production mode of the Higgs on ATLAS.

\section{Topology of VBF $\HWW$ production}

As discussed in Chapter 1, the characteristic feature of VBF production of the Higgs is the presence of two additional forward jets coming from the incoming partons which radiate the vector bosons that make the Higgs. These jets are forward because the outgoing partons still carry the longitudinal momentum of the incoming partons. Figure~\ref{fig:VBF_LeadJetEta} shows the distribution of the $\eta$ for the leading jet in a VBF event compared to a background top pair production event. As can be seen, the VBF jets tend to be more forward in $\eta$, while the $\ttbar$ jets are more central. 

\begin{figure}
  \vspace{20pt}
  \centering
  \hspace*{-32pt}
  \includegraphics[width=0.6\textwidth]{figures/VBF_LeadJetEta}
  \caption{Leading jet $\eta$ in VBF $\HWW$ (red) and $\ttbar$ (black)}
  \label{fig:VBF_LeadJetEta}
\end{figure}

Because the cross section for VBF production is about an order of magnitude smaller than gluon fusion production, these forward jets must be used in order to better reduce background and achieve a good signal to background ratio. The analysis selection is constructed to maximally exploit the features of the unique VBF topology. 

\section{Data and simulation samples}

The results presented here are with 20.3 \ifb taken at $\sqrt{s} = 8 \TeV$ and 4.5 \ifb taken at $\sqrt{s} = 7 \TeV$. The details of the LHC and detector conditions during this period are given in Chapter 2. The trigger selection defining the dataset is discussed in section~\ref{sec:HWWtrigger}. The simulation samples used for signal and background modeling are given in section~\ref{sec:HWWMC}.

\subsection{Triggers}
\label{sec:HWWtrigger}

The analysis uses a combination of single lepton and dilepton triggers to allow lowering of the $\pT$ thresholds and increased signal acceptance. As discussed in Chapter 2, there are multiple levels in the ATLAS trigger system, and there are different $\pT$ thresholds imposed for the leptons at each level. Additionally, some triggers have a loose selection on the isolation of the lepton (looser than that applied offline in the analysis object selection). Table~\ref{tab:single-lepton-trig} shows the thresholds used for single lepton triggers, while table~\ref{tab:dilepton-trig} shows the thresholds coming from di-lepton triggers. The single lepton trigger efficiency for muons that pass the analysis object selection is 70\% for muons in the barrel region ($|\eta| < 1.05$) and 90\% in the endcap region. The electron trigger efficiency increases with electron $\pT$ but the average is approximately 90\%. These efficiencies are measured by combined performance and trigger signature groups\cite{MuonTrigger2012,ElectronTrigger2012}.



\begin{table}[h!]
\centering
\captionsetup{justification=centering}

%\begin{tabular*}{0.480\textwidth}{p{0.075\textwidth} p{0.180\textwidth} l}
\hspace{-10pt}
\begin{tabular}{|c|c|c|}
\hline
 & Level-1 threshold & High-level threshold \\ \hline \hline
\multirow{2}{*}{Electron} & $18$ & $24i$ \\ 
 & $30$ & $60$ \\ \hline

\multirow{2}{*}{Muon} & \multirow{2}{*}{$15$} & $24i$ \\ 
& & $36$ \\ 
 \hline

\end{tabular}

\caption{
Single lepton triggers used for electrons and muons. A logical ``or" of the triggers listed for each lepton type is taken. Units are in GeV, and the $i$ denotes an isolation requirement in the trigger. 
}
\label{tab:single-lepton-trig}
\end{table}

\begin{table}[h!]
\centering
\captionsetup{justification=centering}

%\begin{tabular*}{0.480\textwidth}{p{0.075\textwidth} p{0.180\textwidth} l}
\hspace{-10pt}
\begin{tabular}{|c|c|c|}
\hline
 & Level-1 threshold & High-level threshold \\ \hline \hline
$ee$ & $10$ and $10$ & $12$ and $12$ \\ \hline
$\mu\mu$ & $15$ & $18$ and $8$ \\ \hline
$e\mu$ & $10$ and $6$ & $12$ and $8$ \\ \hline
\end{tabular}

\caption{
Di-lepton triggers used for different flavor combinations. The two thresholds listed refer to leading and sub-leading leptons, respectively. The di-muon trigger only requires a single lepton at level-1. 
}
\label{tab:dilepton-trig}
\end{table}

The combination of all triggers shown gives good efficiency for signal events. This efficiency is summarized in table~\ref{tab:trigeff}. The relative improvement in efficiency by adding the dilepton triggers is also shown in the same table. The largest gain comes in the $\mu\mu$ channel. Overall the trigger selection shows a good efficiency for $\HWW$ signal events.

\begin{table}[h!]
\centering
\captionsetup{justification=centering}

%\begin{tabular*}{0.480\textwidth}{p{0.075\textwidth} p{0.180\textwidth} l}
\hspace{-10pt}
\begin{tabular}{|c|c|c|}
\hline
Channel & Trigger efficiency & Gain from $2\ell$ trigger \\ \hline \hline
$ee$ & $97$\% & $9.1$\% \\ \hline
$\mu\mu$ & $89$\% & $18.5$\% \\ \hline
$e\mu$ & $95$\% & $8.3$\% \\ \hline
$\mu e$ & $81$\% & $8.2$\% \\ \hline


\end{tabular}

\caption{
Trigger efficiency for signal events and relative gain of adding a dilepton trigger on top of the single lepton trigger selection. The first lepton is the leading, while the second is the sub-leading. Efficiencies shown here are for the ggF signal in the $\Njet = 0$ category but are comparable for the VBF signal. 
}
\label{tab:trigeff}
\end{table}



\subsection{Monte Carlo samples}
\label{sec:HWWMC}

Modeling of signal and background processes in the signal region, in particular for the $\mTH$ distribution, is an important consideration for the final interpretation of the analysis. Therefore, careful consideration must be paid to which Monte Carlo (MC) generators are used for specific processes. With the exception of the $W$+jet and multijet backgrounds, the $\mTH$ shape used as the final discriminant is taken from simulation. (Many backgrounds are normalized from data, as described in section~\ref{sec:HWWbkg}).

Table~\ref{tab:HWW-MC} shows the MC generators used for the signal and background processes, as well as their cross sections. In order to include corrections up to next-to-leading order (NLO) in the QCD coupling constant $\alpha_{s}$, the \POWHEG\cite{powheg1} generator is often used. In some cases, only leading order generators like \ACERMC\cite{acermc} and \GGTOVV\cite{gg2vv} are available for the process in question. If the process requires good modeling for very high parton multiplicities, the \SHERPA\cite{sherpa} and \ALPGEN\cite{alpgen} generators are used to provide merged calculations for five or fewer additional partons. These matrix element level calculations must then be additionally matched to models of the underlying event, hadronization, and parton shower. There are four possible generators for this: \SHERPA, \PYTHIA6\cite{pythia6}, \PYTHIA8\cite{pythia8}, or \HERWIG\cite{herwig} + \JIMMY\cite{jimmy}. The simulation additionally requires an input parton distribution function (PDF). The \CT10\cite{ct10} PDFs are used for \SHERPA and \POWHEG simulated samples, while \CTEQ6L1\cite{cteq} is used for \ALPGEN+\HERWIG and \ACERMC simulations. The Drell-Yan samples are reweighted to the \MRST\cite{mrst} PDFs, as these are found to give the best agreement between data and simulation. 

\begin{table}[t!]
\centering
\captionsetup{justification=centering}
\begin{tabular*}{0.75\textwidth}{
    lll p{0.25\textwidth} c
}
\dbline
\multicolumn{3}{l}{\multirow{2}{*}{Process}}
& \multicolumn{1}{l}{\multirow{2}{*}{MC generator$\nq$}}
& \multicolumn{1}{r}{$\sigma{\CDOT}\mathcal{B}$~~}
\\
&
&
&
& \multicolumn{1}{r}{(pb)~~}
\\
\sgline
\multicolumn{2}{l}{Signal }& & \\
\quad ggF    &$\HWW$                                                             && \POWHEG+\PYTHIA8      & 0.435 \\
\quad VBF    &$\HWW$                                                             && \POWHEG+\PYTHIA8      & 0.0356 \\
\quad $\VH$  &$\HWW$                                                             && \PYTHIA8              & 0.0253 \\
\sgline
\multicolumn{3}{l}{$\WW$ }& & \\
\multicolumn{3}{l}{\quad $\qq{\TO}\WW$ and $qg{\TO}\WW$                          }& \POWHEG+\PYTHIA6      & 5.68 \\ 
\multicolumn{3}{l}{\quad $gg{\TO}\WW$                                            }& \GGTOVV+\HERWIG       & 0.196 \\
\multicolumn{3}{l}{\quad $(\qq{\TO}W){\PLUS}(\qq{\TO}W)$                         }& \PYTHIA8              & 0.480 \\
\multicolumn{3}{l}{\quad $\qq{\TO}\WW$                                           }& \SHERPA               & 5.68 \\
\multicolumn{3}{l}{\quad VBS $\WW{+\,}2\,\textrm{jets}$                          }& \SHERPA               & 0.0397 \\
\sgline
\multicolumn{3}{l}{Top quarks }& & \\
\multicolumn{3}{l}{\quad $\ttbar$                                                }& \POWHEG+\PYTHIA6      & 26.6 \\
\multicolumn{3}{l}{\quad $Wt$                                                    }& \POWHEG+\PYTHIA6      & 2.35 \\
\multicolumn{3}{l}{\quad $tq\bar{b}$                                             }& \ACERMC+\PYTHIA6      & 28.4 \\
\multicolumn{3}{l}{\quad $t\bar{b}$                                              }& \POWHEG+\PYTHIA6      & 1.82 \\
\sgline
\multicolumn{3}{l}{Other dibosons ($VV$)}& & \\
\multicolumn{1}{l}{\quad $\Wg$  } &\multicolumn{2}{l}{($\pT^{\gamma}{\GT}8\GeV$) }& \ALPGEN+\HERWIG       & 369 \\
\multicolumn{1}{l}{\quad $\Wgs$ } &\multicolumn{2}{l}{($\mll{\LE}7\GeV$)         }& \SHERPA               & 12.2 \\ 
\multicolumn{1}{l}{\quad $\WZ$  } &\multicolumn{2}{l}{($\mll{\GT}7\GeV$)         }& \POWHEG+\PYTHIA8      & 12.7 \\ 
%\multicolumn{2}{l}{\quad $\WZ{\PLUS}2\,\textrm{jets}$, ${\cal{O}}(\alpha_s^0)$} &($\mll{\GT}7\GeV$) & \SHERPA  & 0.013 \\
\multicolumn{3}{l}{\quad VBS $\WZ{\PLUS}2\,\textrm{jets}$                        }& \SHERPA               & 0.0126 \\
\multicolumn{1}{l}{\quad        } & ($\mll{\GT}7\GeV$)                            &                       & \\
\multicolumn{1}{l}{\quad $\Zg$  } &\multicolumn{2}{l}{($\pT^{\gamma}{\GT}8\GeV$) }& \SHERPA               & 163 \\
\multicolumn{1}{l}{\quad $\Zgs$ } &\multicolumn{2}{l}{(min.\ $\mll{\LE}4\GeV$)   }& \SHERPA               & 7.31 \\
\multicolumn{1}{l}{\quad $\ZZ$  } &\multicolumn{2}{l}{($\mll{\GT}4\GeV$)         }& \POWHEG+\PYTHIA8      & 0.733 \\
\multicolumn{3}{l}{\quad $\ZZ{\TO}\ell\ell\,\nu\nu$ ($\mll{\GT}4\GeV$)           }& \POWHEG+\PYTHIA8      & 0.504 \\
\sgline
\multicolumn{3}{l}{Drell-Yan }& & \\
\multicolumn{1}{l}{\quad $Z$   } &\multicolumn{2}{l}{($\mll{\GT}10\GeV$)         }& \ALPGEN+\HERWIG  $\np$& 16500 \\
\multicolumn{3}{l}{\quad VBF $Z{\PLUS}2\,\textrm{jets}$                          }& \SHERPA               & 5.36 \\
\multicolumn{1}{l}{\quad        } & ($\mll{\GT}7\GeV$)                            &                       & \\
\dbline
\end{tabular*}
\caption{
  Monte Carlo samples used to model the signal and background processes\cite{WW2015}.
}
\label{tab:HWW-MC}
\end{table}


Once the basic hard scattering process is simulated, it must be passed through a detector simulation and additional pile-up events must be overlaid. The pile-up events are modeled with \PYTHIA8, and the ATLAS detector is simulated with \GEANT4\cite{geant4}. Because of the unique phase space of the $\HWW$ analysis, events are sometimes filtered at generator level to allow for more efficient generation of relevant events. The efficiency of the trigger in MC simulation does not always match the measured efficiency in data, so trigger scale factors are applied to correct the MC efficiency to the data. These are derived by the combined performance groups\cite{MuonTrigger2012,ElectronTrigger2012}.

\section{Object selection}

In order to define the signal region, the analysis must first select the objects to be considered. The details of the object reconstruction algorithms are discussed in Chapter 2, while this section gives specific selection cuts used in the $\HWW$ analysis. 

The first step in this process is to select a primary vertex candidates. The event's primary vertex is the vertex with the largest sum of $\pT^2$ for associated tracks and is required to have at least three tracks with $\pT > 450$ \MeV. Many of the object selection cuts are then made relative to this chosen primary vertex.

\subsection{Muons}

The analysis uses combined muon candidates, where a track in the Inner Detector has been matched to a standalone track in the Muon Spectrometer. The track parameters are combined statistically in the muon reconstruction algorithm\cite{MuonReco}. The muons are required to be within $|\eta| < 2.5$ and have a $\pT > 10 \GeV$. To reduce backgrounds coming from mis-reconstructed leptons, there are requirements on the impact parameter of the muon relative to the primary vertex. The transverse impact parameter $d_0$ is required to be small relative to its estimated uncertainty, the exact cut value being $d_0/\sigma_{d_0} < 3$. The longitudinal impact parameter $z_0$ must satisfy $\left|z_0\sin\theta\right| < 1$ mm. 

As discussed previously, the muons must also be isolated. There are two types of lepton isolations that are calculated: track-based and calorimeter-based. For muons, the track-based isolation is defined using the scalar sum  $\sum \pT$ for tracks with $\pT > 1 \GeV$ (excluding the muon's track) within a cone of  $\Delta R = 0.3$ ($0.4$) for muon with $\pT > 15 \GeV$ ($10 < \pT < 15 \GeV$). The final isolation requirement is made my requiring that this scalar sum be no more than a certain fraction of the muon's $\pT$. This requirement varies with muon $\pT$ and the exact cuts are defined in table~\ref{tab:muonisocuts}.

The calorimeter-based muon isolation is defined using as a $\sum E_{T}$ calculated from calorimeter cells using the same cone size as the track-based isolation but excluding cells with $\Delta R < 0.05$ around the muon. This requirement is also definedas a cut on the ratio of the sum to the muon $\pT$ and varies with muon $\pT$. The cut values are also given in table~\ref{tab:muonisocuts}.

The isolation requirements loosen as a function of $\pT$ to allow for larger signal acceptance. At low $\pT$, the isolation is tightened to reduce the $W$+jets background which arises from a misidentified lepton. 

\begin{table}[h!]
\centering
\captionsetup{justification=centering}

%\begin{tabular*}{0.480\textwidth}{p{0.075\textwidth} p{0.180\textwidth} l}
\hspace{-10pt}
\begin{tabular}{|c|c|c|}
\hline
$p_T$ range (GeV) & Calorimeter isolation & Track isolation\\ \hline \hline
$10-15$ & $0.06$ & $0.06$ \\ \hline
$15-20$ & $0.12$ & $0.08$ \\ \hline
$20-25$ & $0.18$ & $0.12$ \\ \hline
$> 25$ & $0.30$ & $0.12$ \\ \hline
\end{tabular}

\caption{
$\pT$ dependent isolation requirements for muons. Muons are required to have the amount of calorimeter or track based cone sums be less than this fraction of their $\pT$.
}
\label{tab:muonisocuts}
\end{table}

\subsection{Electrons}

Electrons are identified by matching reconstructed clusters in the electromagnetic calorimeter with tracks in the inner detector. The electrons are identified using a likehood based method\cite{ElectronReco,GSF} which takes into account the shower shapes in the calorimeter, the matching of tracks to clusters, and the amount of transition radiation in the TRT. The electrons are required to have $|\eta| < 2.47$, and candidates in the transition region between the barrel and endcap ($1.37 < |\eta| < 1.52$) are excluded. As the muons, the electrons are required to have transverse impact parameter significance $ < 3$, while in the longitudinal direction they must have $|z_0 \sin \theta| < 0.4$ mm. Some electron requirements also vary with electron $E_{T}$, and these requirements are summarized in table~\ref{tab:elecselec}.

The isolation for electrons are defined similarly to the muons but with unique cuts on the objects included. The track-based isolation is defined using tracks with $\pT > 400 \MeV$ with cone sizes as defined previously. The calorimeter-based isolation also uses the same cone size as the muon, but here the cells within a $0.125 \times 0.175$ area in $\eta \times \phi$ around the electron cluster's barycenter are excluded. The other difference with respect to muons is that the denominator of the isolation ratio is the electron's $E_{T}$ rather than $\pT$. The isolation cuts very with electron $E_{T}$and are defined in table~\ref{tab:elecselec}. 

The electron is also required to not be consistent with a vertex coming from a photon conversion. 


\begin{table}[h!]
\centering
\captionsetup{justification=centering}

%\begin{tabular*}{0.480\textwidth}{p{0.075\textwidth} p{0.180\textwidth} l}
\hspace{-10pt}
\begin{tabular}{|c|c|c|c|}
\hline
$p_T$ range (GeV) & Quality cut & Calorimeter isolation & Track isolation\\ \hline \hline
$10-15$ & Very tight LH & $0.20$ & $0.06$ \\ \hline
$15-20$ & Very tight LH & $0.24$ & $0.08$ \\ \hline
$20-25$ & Very tight LH & $0.28$ & $0.10$ \\ \hline
$> 25$ & Medium & $0.28$ & $0.10$ \\ \hline
\end{tabular}

\caption{
$\pT$ dependent requirements for electrons. Electrons are required to have the amount of calorimeter or track based cone sums be less than this fraction of their $E_{T}$.
}
\label{tab:elecselec}
\end{table}

\subsection{Jets}

Jets are clustered with the anti-$k_T$ reconstruction algorithm using a radius parameter of $R=0.4$. They are required to have a jet vertex fraction (\jvf) of at least 50\%, meaning that half of the tracks associated with the jet originated from the primary vertex. Jets with no tracks associated (i.e. those outside the acceptance of the ID) do not have this requirement applied. Jets are required to have $p_{T} > 25 \GeV$ if they are within the tracking acceptance ($|\eta| < 2.4$). Jets with $2.4 < |\eta| < 4.5$ are required to have $\pT > 30 \GeV$. This tighter requirement reduces jets from pileup in the region where \jvf requirements cannot be applied. The two highest $\pT$ jets in the event are referred to as the ``VBF" jets and used to compute various analysis selections later. 

Identification of $b$-jets is done using the MV1 algorithm and is limited to the acceptance of the ID ($|\eta| < 2.5$). The operating point of MV1 that is used is the one that is 85\% efficient for identifying true $b$-jets. This operating point has a 10.3\% of mis-tagging a light quark jet as a $b$-jet. In order to improve the rejection of $b$-jets, a lower threshold than the nominal $\pT$ threshold described above is used. For the purposes of counting the number of $b$-jets, jets with $\pT$ down to 20 \GeV are used. 

\subsection{Overlap removal}

There are some cases where certain reconstructed objects will overlap and one will have to be chosen (for example, an electron and a jet in the calorimeter). First, the case of lepton overlap is dealt with. If an electron candidate extends into the muon spectrometer, it is removed. If a muon or electron have a $\Delta R < 0.1$, the electron is removed and the muon is kept. If two electron candidates overlap within the same radius, then the higher $E_{T}$ electron is kept. Next, the overlap between leptons and jets is considered. If an electron and jet are within $\Delta R < 0.3$ of one another, the electron is kept and the jet is removed. However, if a muon and jet overlap within $\Delta R < 0.3$, the jet is kept (as it is likely that the muon is the result of a semileptonic decay inside the jet). 

Once the overlap removal is complete, the final set of objects used in the analysis is defined. 


\section{Analysis selection}

The VBF analysis uses two distinct selections. The first is a looser selection that uses a Boosted Decision Tree (BDT) score as the final discriminator in order to take advantage of the detailed correlations between the VBF variables. The second is a more standard selection, referred to as ``cut-based", that applies cuts on the VBF variables and uses $\mTH$ as the final discriminating variable. While the BDT analysis is ultimately more sensitive, the cut-based serves as an important component of the analysis. First, the cut-based allows for confirming the modeling and validity of many variables used as input to the BDT. Second, because this is the first use of such an MVA technique in the $\HWW$ analysis, the cut-based selection allows confirmation of the final BDT result with a more traditional analysis. Both analyses will be discussed here. 

One important note is that because this analysis is dedicated to the measurement of the VBF production mode of the Higs, events coming from gluon fusion production with the Higgs decaying via \HWWfull are treated as background events. This will be seen throughout the vargious cutflow tables and yields shown. 

\subsection{Common pre-selection}

Both the BDT and cut-based analyses have a common pre-selection that is applied before their main cuts. The cuts on leptons are common to all $\Njet$ bins. The analysis requires two oppositely charged leptons, with the leading lepton required to have $\pT > 22$ \GeV while the subleading lepton must have $\pT > 10$ \GeV. Next, to cut out low mass $\ZDY$ events, a cut on the dilepton mass $\mll > 10$ (12) \GeV is applied in the different (same) flavor channel. In the same flavor channels, there is an additional veto placed on the region around the Z peak, requiring that $|\mll - \mZ| > 15$ \GeV. 

Finally, there are requirements on the amount of missing transverse momentum in the event. These are only applied in the same flavor channels, as in the different flavor channels $\ttbar$ is the dominant background in $\Njet \geq 2$. The BDT analysis requires $\MPT > 40 \GeV$ and $\MET > 45 \GeV$. The cut-based analysis must cut tighter on these variables to have maximal sensitivity and thus requires $\MPT > 50 \GeV$ and $\MET > 55 \GeV$. 


%In the different flavor channels, a similar cut is made to reduce the contamination from $Z\to\tau\tau$ decays. The di-$\tau$ invariant mass, $\mtt$. is constructed by assuming that the neutrinos from the $\tau$ decays were collinear with the leptons\cite{collinear}. The analysis requires that this mass not be consistent with a $Z$ by requiring $\mtt < mZ - 25$ \GeV. Finally, because this analysis is focused on VBF, a requirement on the jet multiplicity is placed, with $\Njet \geq 2$.

\subsection{Cut-based selection}

The cut-based selection places sequential requirements on variables reconstructed from the VBF jets in order to increase the signal to background ratio.  


\section{Background estimation}
\label{sec:HWWbkg}

\section{Systematic uncertainties}

\section{Results}
